While the correctness of an implementation with respect to a formal functional specification provides a very strong form of guarantee, it can be very costly to achieve.
Currently it is mostly reserved to domains where it is required by regulations or offers a competitive advantage.
In practice, it is very useful to rely on a combination of formal methods to achieve an appropriate degree of guarantee:
automatic static analysis to ensure the absence of runtime errors,
deductive verification to prove functional correctness,
and runtime verification for parts of code that cannot be (or are not yet) proved using deductive verification,
or parts of code that contain \emph{warnings} from static analysis requiring confirmation.

This project will use \framac{} (\url{https://frama-c.com})~\cite{KKP2015:FAC}.
It is a widely-used source code analysis platform that aims at conducting verification of industrial-size programs written in ISO C99 source code.
\framac{} fully supports combinations of different approaches, by providing its users with a collection of \emph{plugins} for static and dynamic analyses of safety- and security-critical software.
Moreover, collaborative verification across cooperating plugins is enabled by their integration on top of a shared kernel, and their compliance to a common specification language: \acsl~\cite{ACSL}.
\acsl, for ANSI/ISO C Specification Language, is based on the notion of contract like in JML.
It allows users to specify functional properties of programs through pre/post-condition, and provides different ways to define predicates and logic functions.
Some useful built-in predicates and logic functions are provided, to handle for example pointer validity or separation.
\framac is very appropriate for the verification of typical legacy embedded C code.
Its specification language is rich but easy to understand: ACSL is essentially a typed first-order logic that contains C expressions.

\emph{Value analysis} is a program analysis technique that computes a set of possible values for every program variable at each program point.
It is based on the \emph{abstract interpretation} technique proposed by Cousot and Cousot in the 1970's~\cite{cousot77}.
Its main idea is to compute an abstract view of values of variables in the form of \emph{abstract domains}.
For example, a usual abstract view for a number value is an interval.
Value analysis can be very useful to detect potential runtime errors or prove their absence.
Typical examples include invalid pointers, invalid array indices, arithmetic overflows or division by zero.
It can also help to prove other properties for which domain-based reasoning can be efficient.
The \Eva (Evolved Value Analysis) plugin is strongly integrated into the \framac ecosystem.
It offers a basis for many other derived plugins (see~\cite{KKP2015:FAC}).

\Wp is a \emph{deductive verification} plugin provided with \framac.
It is based on a weakest precondition calculus.
Given a C program annotated in \acsl, \Wp generates the corresponding proof obligations that can be discharged by SMT solvers or with interactive proof.
A combination of automatic and interactive proofs often offers a good trade-off for a complete proof.
Indeed, some properties can only be defined recursively, and in this case, SMT solvers often become inefficient, trying to unroll them.
By using inductive or axiomatically defined functions, we can prevent this behavior but reasoning about them still requires induction, a task that SMT solvers are not good at.
Thus, the last step is generally to state lemmas that can be directly instantiated by SMT solvers.
This last step hinders the adoption of \framac as it requires the users to also master the \Coq proof assistant or another interactive theorem prover.
Our recent work showed how to avoid using an interactive theorem prover for this last step~\cite{BLK2019:NFM}.
Function contracts in ACSL and loop annotations (verified using SMT solvers) are used instead of ACSL lemmas and \Coq proof scripts.
This strengthen the goal of this project: to avoid developers to learn programming paradigms or languages.

\framac was initially designed as a static analysis platform, but it was later extended with plugins for dynamic analysis.
One of these plugins is \eacsl, a runtime verification tool.
\eacsl supports runtime assertion checking~\cite{CR2006:SEN}.
Assertions are very convenient for detecting errors and providing information about their locations.
It is the case even when such an error does not result in a failure during execution.
In \framac, \eacsl is both the name of the assertion language and the name of a plugin that generates C code to check these assertions at runtime.
\eacsl is a subset of \acsl: the specifications written in this subset can therefore be used both by \Wp and \eacsl.
\Wp tries to prove the correctness of these assertions {\em statically} using automated provers, while the plugin \eacsl is used to translate these assertions into C code that can then be executed.
In this case the assertions are checked {\em dynamically}.