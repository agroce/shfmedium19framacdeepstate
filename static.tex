Ideally, the development of critical embedded systems should rely on a combination of formal methods to achieve an appropriate degree of guarantee:
automatic static analysis to ensure the absence of some runtime errors,
deductive verification to prove functional correctness of aspects of the code,
and runtime verification for parts of the code that cannot be (or are not yet) proved using deductive verification,
or that generated \emph{warnings} from static analysis requiring confirmation of a problem or refutation of the feasibility of an error.  A core question is how to represent a specification that is intimately tied to real implementation code.  

This project will therefore make use of the ideas developed in the ongoing work on the \framac{} (\url{https://frama-c.com})~\cite{KKP2015:FAC} tool as a foundation for a specification and annotation language.
\framac is a widely-used source code analysis platform that aims at enabling verification of industrial-size programs,
 supports combinations of different approaches, by providing its users with a collection of \emph{plugins} for analyses.
Moreover, collaborative verification across cooperating plugins is enabled by their integration on top of a shared kernel, and, most critically for our purposes, their compliance to a common specification language: \acsl~\cite{ACSL}.
\acsl, the ANSI/ISO C Specification Language, is based on the notion of a contract as in JML~\cite{jml}: \acsl allows users to specify functional properties of programs through pre/post-conditions.
Many built-in predicates and logic functions are provided, to handle, for example, pointer validity and separation.
Using \acsl/\framac means that while focusing on dynamic analysis and model checking, our approach automatically provides an additional benefit for embedded systems engineers: access to the current set of powerful \framac static analyses.  \framac provides both abstract interpretation~\cite{cousot77} based analysis plugins and deductive verification plugins based on a weakest precondition calculus; the latter have recently been improved to make proof without interacting with a theorem prover easier for engineers~\cite{BLK2019:NFM}.

\framac was designed as a static analysis platform, but has been extended with limited plugins for dynamic analysis.
One of these plugins is \eacsl, which supports runtime assertion checking~\cite{CR2006:SEN}.
In \framac, \eacsl is both the name of the assertion language and the name of a plugin that generates C code to check these assertions at runtime.
\eacsl is a subset of \acsl.  The plugin \eacsl is used to translate a subset of \framac assertions into executable C code.
However, the \eacsl plugin does not support all the specification constructs we need, or assist developers in the most difficult part of dynamic analysis:  constructing a set of tests that exercise the checks.  The only assistance provided by \framac for this is very limited in capability.  Rather than ``re-inventing the wheel'' and offering a solution that lacks a strong static aspect we therefore extend \acsl and \eacsl.