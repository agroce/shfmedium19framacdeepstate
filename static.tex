Ideally, the development of critical embedded systems should rely on a combination of formal methods to achieve an appropriate degree of guarantee:
automatic static analysis to ensure the absence of some runtime errors,
deductive verification to prove functional correctness of aspects of the code,
and runtime verification for parts of the code that cannot be (or are not yet) proved using deductive verification,
or that generated \emph{warnings} from static analysis requiring confirmation of a problem or refutation of the feasibility of an error.  A core question in our approach is how to represent a specification in a way that is intimately tied to real implementation code.  However, as development of a specification language is not our primary focus, we prefer to build on existing work in the area, extending it as required.

This project will therefore make use of the ideas developed in the ongoing work on the \framac{} (\url{https://frama-c.com})~\cite{KKP2015:FAC} tool as a foundation for a specification and annotation language.
\framac is a widely-used source code analysis platform that aims at enabling verification of industrial-size programs written in ISO C99.
\framac{} supports combinations of different approaches, by providing its users with a collection of \emph{plugins} for static and dynamic analyses of safety- and security-critical software.
Moreover, collaborative verification across cooperating plugins is enabled by their integration on top of a shared kernel, and, most critically for our purposes, their compliance to a common specification language: \acsl~\cite{ACSL}.
\acsl, the ANSI/ISO C Specification Language, is based on the notion of a contract as in JML, the Java Modeling Language~\cite{jml}.
\acsl allows users to specify functional properties of programs through pre/post-conditions, and provides different ways to define predicates and logic functions.
Many built-in predicates and logic functions are provided, to handle, for example, pointer validity or separation; ACSL is essentially a typed first-order logic that contains C expressions, and is designed to be rich but easy to understand.
\framac is already applicable to verification of typical legacy embedded C code.

Using \acsl/\framac means that while focusing on dynamic analysis and model checking, our approach automatically provides an additional benefit for embedded systems engineers: access to the current set of \framac analyses, which are themselves powerful and useful.  For example, \framac provides advanced \emph{value analysis} plugins, based on \emph{abstract interpretation}~\cite{cousot77}, where abstracting variable values, e.g., as intervals, can detect invalid pointers, invalid array indices, arithmetic overflows or divisions by zero.  The \Eva (Evolved Value Analysis) plugin is well integrated with the \framac ecosystem, 
and is the basis for many other derived plugins (see~\cite{KKP2015:FAC}).
Another \framac plugin, \Wp, provides \emph{deductive verification} based on a weakest precondition calculus.
Given a C program annotated in \acsl, \Wp generates the corresponding proof obligations that can be discharged by SMT solvers or via interactive proof.
Recent work  showed how to avoid using an interactive theorem prover in some cases even for inductive lemmas~\cite{BLK2019:NFM}, using 
function contracts in ACSL and loop annotations (verified using SMT solvers), making it more possible for engineers to produce some proofs.

\framac was initially designed as a static analysis platform, but it was later extended with (very limited) plugins for dynamic analysis.
One of these plugins is \eacsl, a runtime verification tool.
\eacsl supports runtime assertion checking~\cite{CR2006:SEN}.
In \framac, \eacsl is both the name of the assertion language and the name of a plugin that generates C code to check these assertions at runtime.
\eacsl is a subset of \acsl: the specifications written in this subset can therefore be used both by \Wp and \eacsl.
\Wp tries to prove the correctness of these assertions {\em statically} using automated provers, while the plugin \eacsl is used to translate these assertions into executable C code.
In this case the assertions are checked {\em dynamically}.

However, the \eacsl plugin does not assist developers in the most difficult part of dynamic analysis:  constructing a set of tests that exercise the checks to find as many bugs as possible or build confidence in the correctness of the system.  The only assistance provided by \framac in this task is, as discussed in the next section, very limited in capability.  Again, rather than ``re-inventing the wheel'' and offering a solution that lacks a strong static aspect we aim to address this limitation, and extend \acsl and \eacsl, in accordance with our focus on common frameworks and cross-tool synergy. 