It is notoriously difficult to design correct and secure communication
protocols. One of the most famous example is the Needham Schroeder
Public Key protocol~\cite{NS1978:CACM}: it took 18 years to discover a
flaw in this protocol~\cite{LOW1996:TACAS}, and it was done using
formal methods.  Networks of timed automata are formalisms suitable
for the formalization of protocols.  They are the basis of model
checkers such as \uppaal~\cite{DLL2015:STTT} and
\prism~\cite{KNP2011:CAV} that have been used successfully in the
verification of network
protocols~\cite{ZBW2013:ENTCS,HMJ2006:MASCOTS,HSS2010:NFM,KPK2015:VECOS}.
These tools help in finding issues with an abstract formulation of a
protocol, but they cannot help with implementation details that are
not directly modeled in the formalism.

But the {\em implementation} details of a correct protocol also matter
as the Heartbleed vulnerability in the OpenSSL implementation
shows. In the case of Heartbleed, the problem was a C runtime error:
an access to an invalid memory region, and it was due to an implicit
assumption on the input of a function that was actually false.
Combination of static and dynamic analyses can detect such
vulnerabilities~\cite{KKP2015:HVC} because they are not due to complex
interactions related to the protocol. 

The methodology we envision combines the use of model-checkers such as
\uppaal and \prism for the verification of protocols, with frameworks
for static and dynamic analysis of C programs, namely \framac and
\deepstate, for the verification of the {\em implementations} of
protocols.

We consider \framac, and its specification language \acsl pivotal in this
approach. The research challenge here is to translate networks of
timed automata into \acsl annotations and C ghost code for enabling
the verification of the C code implementing the protocol modeled by
the timed automata. This problem can further be broken down into the
following key-subproblems:

\begin{enumerate}[labelsep=3pt,leftmargin=12pt]
\item The translation of networks of automata into annotations to be
  used within the \framac code analyzer. Previous work co-authored by
  co-PI Loulergue on the Contiki~\cite{DGV2004:LCN} lightweight
  operating system for the Internet of Things showed that various
  approaches can be applied to the verification of the same
  code. For checking the correctness of the linked list API of
  Contiki, it includes the use of ghost arrays~\cite{BKL2018:NFM}.
  Ghost code is a part of a program that is added for the purpose of
  specification. Such code should not interfere with regular
  code. Erasing it should make no observable difference in the program
  results. This approach made it possible to perform most proofs
  automatically using the \framac/\Wp tool, only a small number of
  auxiliary lemmas being proved interactively in the \Coq proof
  assistant.  This work relied on an elegant segment-based reasoning
  over the companion array developed for the proof.  
  
  This approach, however, is expressed in parts of the \acsl language that cannot be
  translated to executable C code, i.e. that do not belong to the
  \eacsl subset. In a broader verification context, especially
  as long as the whole system is not yet formally verified, it is very
  useful to rely on runtime verification, in particular to test client
  modules that use the list module. Another work~\cite{LBK2018:TAP}
  showed a variant of the list module specification that belongs to
  the executable subset \eacsl of \acsl and can be transformed into
  executable C code. A newer approach~\cite{BKL2019:SAC} relies on
  logic lists: they are part of the \acsl standard library of
  inductively defined logical data structures. In the case of Contiki,
  a logic list provides a convenient high-level view of the linked
  list.  The specifications of all functions are now proved faster and
  almost all automatically, only a small number of auxiliary lemmas
  and a couple of assertions being proved interactively in \Coq.
  
  We expect several translations of networks of automata to be
  considered: some may be easier to understand to C programmers not
  familiar with formal specifications (ghost code), some more efficient for
  deductive verification (logical data structures), and some suited
  for dynamic verification while still being amenable to deductive
  verification.

  Such translations will be implemented as \framac plugins. The
  alphabet (events) and states of the automata will need to be
  associated with respectively specific execution events and memory
  states of the C programs. We expect to experiment manually with this
  mapping in case studies before enhancing the plugin to provide
  support for it.

\item Although deductive verification about {\em algorithmic
    complexity} is possible from source
  code~\cite{WK2009:TYPES,PS2014:SAC,GCP2018:ESOP}, such a formal way
  is not appropriate for this project, essentially because these
  approaches deal with complexity rather than execution time, and
  there is no precise enough translation from one to the other.

  Time constraints will be translated into new \acsl annotations.
  Then we will rely on static analysis tools for worst-case execution
  time estimation. There are recent projects~\cite{MRP2017:WCET} that
  explore taking advantage of semantics information to improve WCET
  estimation, as well as preliminary work showing the benefits of such
  an approach~\cite{BA2014:JRWRTC}.

  A C program with \acsl annotations very often provides a large
  variety of semantic information including intervals for variable
  values, and information about loops such as relations between the
  number of iterations and other variables. Exploiting this
  information will require us to be able to modify the WCET tool.  This
  requirement excludes the best WCET estimator,
  aiT~\cite{FER2004:IPDPS}, a closed source commercial tool.
  Heptane~\cite{HRP2017:WCET} and {\tt OTAWA}~\cite{BCR2010:SEUS} are
  two actively developed open source projects with software
  architectures designed to ease extending the tools. Heptane is
  focused on cache analysis while {\tt OTAWA} supports more processor
  architectures. For our case study, {\tt OTAWA} seems the best
  alternative for verifying if the bound obtained by WCET estimation on
  the code indeed satisfies the time constraints obtained for the timed
  automata.

\item The main part of the code where the specification of the
  automata will be used should be a kind of event loop. However this
  loop may be incomplete in the sense that it may not consider all the
  possible events, or even may not be structured as an event loop in the case
  most events are handled through interrupts. Although \framac does not directly
  handle concurrency, its \conctoseq~\cite{BKL2016:SCAM} plugin allows for the 
  analysis of parallel compositions of C programs through program transformation to sequential C
  programs~\cite{BLK2017:VPT}. The main part of the simulating program
  is a loop that handles control switch among the various
  threads. This loop can be used as the main event loop in a 
  concurrency context.
  
\end{enumerate}



% ;;; Local Variables: ***
% ;;; mode: latex ***
% ;;; eval: (ispell-change-dictionary "english" nil) ***
% ;;; eval: (flyspell-buffer) ***
% ;;; End: ***
