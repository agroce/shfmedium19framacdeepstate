\section{Results From Prior NSF Support}

\paragraph{PI Groce:}
The most relevant prior NSF support for PI Groce is CCF-
CCF-2129446, ``Feedback-Driven Mutation Testing for Any Language,'' with a total budget of \$500,000 from 9/2021 until 8/2024,
a collaborative proposal with Claire Le Goues of Carnegie Mellon
University. {\bf Intellectual Merit:} This project
stages, focuses on a synergistic approach for allowing real developers,
particularly for embedded and critical systems,
to improve testing by using mutation testing (both to identify
weaknesses in tests and to directly test systems).  Though in its first
year,  this project has already resulted in two
publications~\cite{cc2022,seip2022}. {\bf
  Broader
  Impact:}  Work from this project has already
resulted in the reporting and correction of mulitple bugs in software
systems, including production compilers for smart contracts, and the
release of a tool for compiler fuzzing.

{\paragraph{Co-PI Flikkema and co-PI Nghiem} are PI and co-PI,
respectively, on the Distributed Sensing \& Computing Over Sparse
Environments (DISCOVER) Platform funded by an NSF CCRI grant
(2120485), with a total budget of \$1,366,513 from 10/2021 until
9/2024.  {\bf Intellectual Merit:} DISCOVER is a cyberinfrastructure
testbed for remote, rural, and sparsely populated areas, consisting of
a fabric of highly configurable and fixed and mobile
Internet-of-Things nodes and a heterogeneous wireless network.  It
will enable focused research on new breeds of algorithms that address
a range of challenges around prioritization and optimization of
computation and communication in remote, less populous and rural
areas.  {\bf Broader Impact:} DISCOVER will provide a research
platform for investigation of distributed computing, networking,
security, control and coordination solutions in a heterogeneous
configurable cyber-physical system infrastructure that will provide
critical services for areas and populations at increasing risk of
being underserved. The education and outreach impacts of this project
include training and research opportunities for undergraduate
students, engaging underrepresented minority students, and developing
hands-on research experiments for K-12 students.

% Flikkema is co-PI on the Southwest Experimental Garden Array (SEGA)
% funded by an NSF development MRI
% (DEB-1126840), with a total budget of \$2,848,876 from 10/2011 until
% 9/2017. {\bf Intellectual Merit:} SEGA is a facility
% distributed across a 1615m elevation gradient in Arizona that supports
% long-term research to increase understanding of and mitigate climate
% change using knowledge of genetic variation in species of concern. It
% consists an array of eleven gardens and supporting distributed
% monitoring and control cyberinfrastructure for the study of
% gene-by-environment interactions and enabling development of
% strategies to best manage for future climates. {\bf Broader
%   Impact:}  With 9 successfully completed projects to date, SEGA
% currently supports 11 experiments and has resulted in over 35
% publications and 20 conference presentations.  SEGA results are
% available online~\cite{SEGA}.


%%% Local Variables:
%%% mode: latex
%%% TeX-master: "main"
%%% End:
