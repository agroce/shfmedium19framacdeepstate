\subsubsection{Communication Protocol for Wireless Sensor Nodes in SEGA}
\label{sec:sega-case-study}

\paragraph{Overview:}

SEGA is a large collection of operational wireless sensor/actuator networks for monitoring and control of ecological systems, located at 17 sites in the states of Arizona and California.
Currently, SEGA consists of 138 wireless nodes and is planned to expand to a total of 154 nodes at 21 sites in the coming years.
As a genetics-based climate change research platform, SEGA allows scientists to quantify the ecological and evolutionary responses of species to changing climate conditions.
Multiple long-term and large-scale scientific experiments are conducted at SEGA sites.
The SEGA project was led by NAU, and the group of co-PI Flikkema developed and are maintaining the wireless nodes, including their embedded software.
%
The SEGA nodes use a multi-processor architecture, in which a central processor provides OS-level services %, including scheduling and dispatch of tasks, storage, and a message-passing interface for wireless networking.
while plug-in satellite processors handle transducer sampling, actuation, and related computational tasks.
In addition to allowing true parallelism, this architecture enables hardware-level improvements in energy efficiency, since each satellite can be optimized for its specific task.
\deleted{More practically, it admits the rapid implementation of highly heterogeneous nodes that incorporate a wide range of sensing and actuation capabilities.}
% The current implementation of the architecture emphasizes energy efficiency~\cite{FliSENSORS2010,FliICC2011}.
% For example, all satellite processors are power-gated via central processor control; ensuring that satellite processors are depowered prevents satellite sleep-mode energy leakage.
% The power subsystem provides multiple power buses at different voltages, including an optically-isolated high-power bus for actuation.
% A variety of energy supplies are also supported, including battery-backed photo-voltaic sources~\cite{FliEtAl12,KnaFli17}.
%
The nodes synchronously interact with neighbors in a multi-hop, self-organizing/healing network, implemented by a custom communication protocol designed by the group of co-PI Flikkema. %; synchronization is implemented as scheduled rendezvous in time slots; slot boundaries are managed by a lightweight global time synchronization protocol that is integrated with low-level communication synchronization.
The nodes use a custom time-triggered RTOS tightly integrated with a time/frequency-hopped PHY/MAC protocol.
This approach %, implemented using a time-triggered architecture on a custom RTOS,
minimizes communication energy cost, which dominates the overall energy consumption.

\paragraph{Challenges:}

Because timing is critical and is determined by the embedded system hardware and software, most testing has occurred at the network level, with extensive in-lab testing with small networks and instrumented field tests.
However, it has been found in long-term deployments % (at dozens of field sites over years of operation)
that occasionally the networking fails and nodes become isolated---we think due to a complex set of subtle bugs rooted in different levels of timing abstraction.
When such a failure occurs, it often spreads from one node to others, causing nodes to seek to rejoin and expend high levels of energy for radio operation and eventually deplete their energy sources.
Eventually, subnets, or sometimes the entire site, are disabled and humans must visit the site to reboot it.
Such failures could %cause scientific data to be lost or invalidated and, even worse, could
affect %damage
or even destroy (e.g., via over-watering), long-running scientific experiments.
%
\deleted{Since access to SEGA installations can be difficult, and in the long run many may be located so remotely that it is cost-prohibitive to send humans to address problems, discovering the source of these in-operation faults, identifying other faults, and generally improving the reliability of the system is critical.}
We aim to use SEGA (in particular the protocol in question and its implementation) as our case study.
This will enable us to apply our approach in a practical setting, and ensure that what we produce is actually usable by engineers of real systems.
%
SEGA is an ideal case study for several reasons.
First, the above mentioned network problem enables exploring how to design, prove, and test time-critical systems in a way that does no harm: human life is not affected in this application, and data is not lost since all sensed information is logged as a local back-up.
On the other hand, reliable operation is important. %, and failure costly. %, because access to manually fix problems can be problematic even with current installations, and in the future this problem will only grow.
Finally, this application uses common data structures for task control blocks, and the operating system at each node schedules and dispatches both periodic and pseudo-randomly scheduled tasks.
Thus the system is representative of %a good example of the
general applications of scheduling and synchronization in time-critical systems written in C.


\paragraph{Plan:}
%%%%%%%%% How will we carry out this case study?

Following our proposed workflow, we will first annotate the implementation with specifications of correctness properties.  We may model the protocol itself as a timed automaton in \uppaal or \prism, in order to ensure that there is not a subtle flaw in the protocol itself, and to model our expectations of behavior in the real system (and to better understand needed specifications).
Either of these steps may expose the source of the mysterious networking failures.
We will use DeepState, driven by harnesses automatically generated by our tools, to generate tests of the implementation components in question, using fuzzing at first, followed by CBMC and SPIN model checking once prototype back-ends are available.
\added{For the purpose of verification and testing, the uncertainty inherent in the physical environment and in wireless communication will be modeled by probabilistic timed automata.}
%DeepState testing may expose faults that are not part of the specification.
% TRUONG: I commented out the following sentences because they are general (why we want to complement \framac with DeepState), therefore they should be included in the general discussion of the workflow.
%
%For instance, using libFuzzer with DeepState we can use LLVM's Undefined Behavior sanitized to catch some classes of undefined behavior that \framac does not take into account.  Furthermore, \framac's ability to prove properties about interactions of multiple functions operating in arbitrary sequence is often limited; such proofs are notoriously hard to construct in general.  DeepState allows us to hope to detect faults when we cannot prove correctness.  DeepState's ability to use symbolic execution as a back-end will be most useful for verifying single functions that are hard to verify with \framac, while state-of-the-art fuzzers will be most useful for sets of functions, or cases where symbolic execution fails to scale.
%
The above workflow will be conducted by an Embedded System Engineering student, who is familiar with the SEGA IoT system but does not have expertise in software verification and testing, using the software tools developed in this project.
Feedback from the engineer in this case study will inform us how to develop and improve the theory and tools for practical usages by non-expert users in real applications.


\subsubsection{\added{Embedded Software of Wireless Sensor Nodes and Robots in DISCOVER}}
\label{sec:case-study:DISCOVER}


\paragraph{Overview:}

\added{%
DISCOVER %(Distributed Sensing and Computing Over Sparse Environments)
is a cyberinfrastructure testbed for remote, rural, and sparsely populated areas.
The project is funded by NSF and led by NAU, whose team includes co-PI Flikkema (PI of DISCOVER) and co-PI Nghiem (co-PI of DISCOVER in charge of robotics).
DISCOVER consists of a fabric of highly configurable Internet-of-Things (IoT) sensor nodes, autonomous and highly capable terrestrial robots and drones, and a heterogeneous wireless network.
DISCOVER sites will be located at the campuses of NAU, Navajo Technical University, and Clemsom University, as well as several remote sites.
The platform will enable focused research in many domains, including data science and machine learning, heterogeneous networked services, distributed computing and AI, control, autonomous robots, and in-network computation, among many others.
%
We will use DISCOVER for two case studies: one on embedded software for wireless sensor nodes and the other on distributed coordination in multi-robot systems.
Coordinated operation of multiple autonomous robots %(multi-robot systems)
has many important real-world applications~\cite{multirobot2005,multirobotsurvey2013}, e.g., in rescue, security, or disaster response missions. %, several autonomous aerial robots can coordinate to survey an area, monitor target objects, % or activities,
%and guide ground robots. % or vehicles.
In such applications, each robot is autonomous but has the capability to coordinate efficiently and safely with other robots to complete a shared mission, often in a distributed manner. % without any central coordinator.
Such coordination is essential in real-world applications where the environment is constantly and unexpectedly changing.
One of the most critical challenges of this application is to guarantee the safety of a coordination plan, which is typically implemented in C code on the embedded computers of the robots and usually involves wireless inter-robot communication, sensing, and actuation.}


\paragraph{Challenges:}

\added{%
As a community research platform, DISCOVER will allow users, who are researchers in relevant field domains, to develop software code and experiments for the DISCOVER stationary nodes (i.e., sensor nodes) and mobile nodes (i.e., terrestrial robots and drones).
A critical step of the process supported by DISCOVER is automatic verification and testing of the embedded code submitted by users for live experiments.
We expect that submitted code is developed by researchers who are not trained in computer science and who do not usually apply best practices in software engineering.
We also expect that submitted code will have a wide spectrum of code quality and may be malicious, either by chance or intentionally.
Automatic functional testing of user code will therefore be of critical importance for the operation and sustainability of DISCOVER.
Another challenge is the fact that the code to be tested and verified is for embedded systems that have highly complex physical dynamics (in the case of robots) and interactions with the physical world and with other physical systems.
Such physical aspects cannot be easily described in code for the purpose of software verification and testing.
Therefore, sophisticated software-based and hardware-in-the-loop simulations, including embedded hardware emulators and robot simulators, are necessary, for which several options will be developed by the DISCOVER team.}

\paragraph{Plan:}

\added{%
  For the wireless sensor nodes, where a major part of their operation is related to data storage and communication, our planned approach will be similar to that of the SEGA case study.
  For the robotic nodes, we will focus on multi-robot coordination since other user errors, such as unauthorized hardware access and unauthorized maneuvers of robots, can be prevented or mitigated through a combination of techniques such as containerization, access rights, and geofencing.
  Our approach will be briefly described below.
  First, we will model a coordination plan %/algorithm
  for multiple robots as a (potentially very complex) network of timed automata.
  Performance specifications will be expressed in temporal logics, e.g., the Signal Temporal Logic (STL)~\cite{donze2010robust}, and checked against the model using verification and testing tools such as \uppaal or S-TaLiRo~\cite{annpureddy2011s}.
  While we do not expect actual user code to be accompanied by formal models, in our case study, this step ensures that the original coordination plan has no subtle flaws, and helps us determine properties that need formulation at the implementation level.
  An implementation of the algorithm in C code, distributed among the robots, will be developed by a robotics/control student.
  The implementation will be annotated with a specification in our extended \acsl/\eacsl.
  We will then use DeepState harnesses to generate tests of the implementation components using fuzzing, symbolic execution, and both bounded SAT/SMT based and explicit-state model checking.
  Finally, we will determine if a timed automata skeleton extracted from the implementation code corresponds to and would help create a full specification such as we developed before beginning implementation.
  % This case study will be conducted by a robotics/control graduate student in the ICONS Lab, using the software tools developed in this project.
  The very different nature and complexity of this study, compared to stationary sensor nodes, will ensure that our methods and tools work in a variety of kinds of real systems.
  % Given the different nature and complexity of this application compared to the SEGA study, the feedback will be much valuable for the development and improvement of the proposed methods and tools for practical usages in a wide spectrum of real systems.
  To overcome the challenge stemming from the complex physical dynamics and interactions of the robots, we will utilize a sophisticated robot simulation environment, based on the Robot Operating System (ROS) \cite{ROS}, with a rich set of predefined scenarios, developed by the DISCOVER team (specifically by the group of co-PI Nghiem).
  An interface between the robot simulation software and the tools developed in this project will be created to enable seamless verification and testing of the robotic code.}


%%% Local Variables:
%%% mode: latex
%%% TeX-master: "main"
%%% End:
