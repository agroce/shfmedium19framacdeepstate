\begin{wrapfigure}[11]{r}{.35\textwidth}
%\begin{figure}[!tp]
%  \centering
  \resizebox{.35\textwidth}{!}{%
  \begin{ganttchart}[%Specs
    hgrid style/.style={black, dotted},
    vgrid, %={*2{black,dotted}, *1{black, dashed},
      %*2{black,dotted}, *1{black, dashed},
      %*2{black,dotted}, *1{black, dashed},
      %*2{black,dotted}, *1{black, solid}},
    x unit=3mm,
    y unit chart=5mm,
    y unit title=5mm,
    %time slot format=isodate,
    title height=1,
    milestone label font=\footnotesize,
    group label font=\bfseries\footnotesize,
    title label font=\bfseries\footnotesize,
    link/.style={->, thick},
    %bar/.style={fill=blue},
    %bar height=0.7,
    %group right shift=0,
    %group top shift=0.7,
    %group height=.3,
    %group peaks width={0.2},
    %inline
    ]{1}{36}
    % labels
    % \gantttitle{A two-years project}{24}\\  % title 1 
    \gantttitle[]{Year 1}{12}                 % title 1
    \gantttitle[]{Year 2}{12}
    \gantttitle[]{Year 3}{12} \\
    \gantttitle{Q1}{3}                      % title 3
    \gantttitle{Q2}{3}
    \gantttitle{Q3}{3}
    \gantttitle{Q4}{3}
    \gantttitle{Q1}{3}
    \gantttitle{Q2}{3}
    \gantttitle{Q3}{3}
    \gantttitle{Q4}{3}
    \gantttitle{Q1}{3}
    \gantttitle{Q2}{3}
    \gantttitle{Q3}{3} 
    \gantttitle{Q4}{3}\\    

    % \ganttgroup[inline=false]{Group 1}{1}{5}\\ 
    % \ganttbar[progress=10,inline=false]{Planning}{1}{4}\\
    % \ganttmilestone[inline=false]{Milestone 1}{9} \\

    % \ganttgroup[inline=false]{Group 2}{6}{12} \\ 
    % \ganttbar[progress=2,inline=false]{test1}{10}{19} \\
    % \ganttmilestone[inline=false]{Milestone 2}{17} \\
    % \ganttbar[progress=5,inline=false]{test2}{11}{20} \\
    % \ganttmilestone[inline=false]{Milestone 3}{22} \\       

    % \ganttgroup[inline=false]{Group 3}{13}{24} \\ 
    % \ganttbar[progress=90,inline=false]{Task A}{13}{15} \\ 
    % \ganttbar[progress=50,inline=false, bar progress label node/.append style={below left= 10pt and 7pt}]{Task B}{13}{24} \\ \\
    % \ganttbar[progress=30,inline=false]{Task C}{15}{16}\\ 
    % \ganttbar[progress=70,inline=false]{Task D}{18}{20} \\ 

    \ganttgroup[
        group/.append style={fill=blue}
    ]{WP1}{1}{36}\\ [grid]
    \ganttbluebar[
        name=T11
    ]{T1.1}{1}{12}\\ [grid]
    \ganttbluebar[
        name=T12
    ]{T1.2}{13}{36}\\ [grid]
    % \ganttlinkedbluebar{}{2014-10-7}{2014-10-10}
    % \ganttlinkedbluebar{}{2014-10-14}{2014-10-15}
    % \ganttlinkedbluebar{}{2014-10-17}{2014-10-17}
    % \ganttlinkedbluebar[name=FMEend]{}{2014-10-21}{2014-10-24}
    % \ganttlinkedbluebar{}{2014-10-28}{2014-10-31}\\ [grid]
    % \ganttbluebar[name=Manual]{Manual}{2014-10-30}{2014-10-31}
    % \ganttlinkedbluebar{}{2014-11-4}{2014-11-7} \ganttnewline[thick, black]

    \ganttgroup[
        group/.append style={fill=blue}
    ]{WP2}{1}{36}\\ [grid]
    \ganttbluebar[
        name=T21
    ]{T2.1}{1}{12}\\ [grid]
    \ganttbluebar[
        name=T22
    ]{T2.2}{13}{36}\\ [grid]

    \ganttgroup[
        group/.append style={fill=blue}
    ]{WP3}{1}{36}\\ [grid]
    \ganttbluebar[
        name=T31
    ]{T3.1}{1}{12}\\ [grid]
    \ganttbluebar[
        name=T32
    ]{T3.2}{13}{36}\\ [grid]
    \ganttbluebar[
        name=T41
    ]{T4.1}{7}{12}\\ [grid]        
    \ganttbluebar[
        name=T42
    ]{T4.2}{13}{36}\\ [grid]
    \ganttbluebar[
        name=T43
    ]{T4.3}{25}{36}    
    % %Implementing links
    % \ganttlink[link mid=0.75]{Documentation}{FME}
    % \ganttlink{FMETutorial}{FME}
  \end{ganttchart}}%
\caption{Project schedule.}%
\label{fig:project-schedule}%
% \end{figure}
%\vspace{-0.4in}
\end{wrapfigure}

The project will be organized into two phases, described by work
packages.  In the first phase, T3.1 will be conducted along with T1.1 and will inform the development in these tasks.
In the second phase, the focus will be on the application of tools in
T1.2 in tandem with T2.
Tasks related to case studies (tasks T4.1, and T4.2), whose results
and feedback will help refine the developed tools will be especially
emphasized in the final phases of the project.
The timeline of the tasks will be structured as shown in Figure~\ref{fig:project-schedule}.

\paragraph{Work Package 1 (WP1):}  This work package concerns the
development of and use of \acsl and \eacsl extensions for use in
embedded system implementation
code.


$\bullet$ T1.1: This task will consider needed extensions for handling
real-world embedded systems.  In particular, there will be a focus on
a study of the formal semantics of timed
automaton networks defined in \uppaal and \prism, to determine the
extent to which shared semantics can be assigned making it possible to
carry implementation annotations into such formal models (see T4.2)
and, in theory, translate properties from such models into implementation annotations.  As such, there will be close collaboration and iterative design steps between this task and the other work packages. %WP3 (Application).

$\bullet$ T1.2: This task will take feedback from applications of
tools to generate tests and proofs (T2) into account, to add annotations
that are focused on heuristic guidance for tools, not correctness per se.

One Ph.D. student will conduct this work, which will last for the entire duration of the project.

\paragraph{Evaluation:} Evaluation of
WP1 will be determined by ability of embedded engineers to agree that
the key properties, including those related to timed automata models, to be checked are (1) all representable by the
annotations (2) easy to construct (3) easy to read when produced by others and
(4) maintainable after introduction.

\paragraph{Work Package 2 (WP2):}  This work package covers
methods and tools to automatically translate \acsl/\eacsl-annotated code in
into a \deepstate test harness (Section~\ref{sec:framac2deepstate}),
the development of DeepState back-ends for CBMC and SPIN, with
appropriate mechanisms to ease the use of these tools, and
improvements to fuzzers to improve test generation:
\begin{itemize}[labelsep=3pt,leftmargin=12pt]
\item T2.1: This task will optimize the implementation of symbolic
  execution and fuzzing in DeepState, so that \acsl/\eacsl annotations
  and extensions from WP1 can be used effectively.
\item T2.2: This task will develop DeepState back-ends for CBMC and
  SPIN, inform annotations needed to handle loop bounds,
  memory tracking and matching, and make use of feedback from fuzzing.
\end{itemize}

The execution of this work package will also span the entire duration of the project.
Because the tasks in this package are also based on developing
verification and test generation tools (thus formal methods
expertise), the same Ph.D. student will work on WP1 and WP2.  We
separate the WPs primarily to emphasize that specification extensions
and tool support are somewhat orthogonal concerns, and evaluated differently.

\paragraph{Evaluation:} Evaluation of
WP2 will be determined by the application
of DeepState harnesses to generate tests for realistic
systems.  We will use benchmarks and simple examples to some
extent, but primarily rely on our connection to case studies.
In the case of test generation, in addition to faults
detected, we will use code coverage and other standard
benchmarks~\cite{FuzzerHicks}.  We expect to publish papers on
advances in fuzzing technology and
fundamental issues arising from the CBMC and SPIN back-ends with
respect to handling loop bounds and memory tracking/matching.

\paragraph{Work Package 3 (WP3):}  This work package will focus on the field applications described in Section~\ref{sec:case-study}, as both a way to inform the methodology and tool developments in the other work packages and case studies %in two completely different domains
to validate our methods and tools.
WP3 includes the following case studies:
\paragraph{Wireless sensor network (WSN) case studies on SEGA and DISCOVER.} This %application
is divided into two tasks:
\noindent  \begin{itemize}[labelsep=3pt,leftmargin=12pt]
\item T3.1: In this task, the %existing SEGA
  wireless sensor node systems will be studied thoroughly to extract
  the key requirements and characteristics of the embedded system
  implementations.  Timed automaton models of the communication
  protocol in each system, at different levels of abstraction, may be
  developed and formally verified in \uppaal and/or \prism, to inform
  task T1.1.  The system information and models resulting from this
  task will inform the semantics design and method developments in WP1
  and WP2.  As time allows we will extend this work to include sensing
  elements.
\item T3.2: This task will apply the tools developed in WP1 and WP2 to
  the WSN systems, %in order
  to detect and fix bugs in
  the %embedded software implementations of the
  communication protocol implementations; in particular, the bugs that
  cause the intermittent failures in
  SEGA. % mentioned in Section~\ref{sec:case-study}.
  It will also provide feedback to the other work packages to refine
  and improve our tools.
  \end{itemize}

\paragraph {Multi-robot system case study on DISCOVER.} This study is divided into three tasks:
\noindent \begin{itemize}[labelsep=3pt,leftmargin=12pt]
\item T4.1: In this task, a standard multi-robot coordination
  algorithm %currently used with our existing multi-robot system
  will be modeled as a network of timed automata.  Using our insights
  into the robotics application, we will express its performance
  specifications, particularly its safety requirements, in temporal
  logics and formally verify or test them in tools like \uppaal,
  \prism, or S-TaLiRo.  This task will extend the developed semantics
  and methods to applications beyond communication protocols, to
  identify further needed runtime extensions and semantic connections
  between timed automata theory, implementation annotations, and
  runtime checks.
\item T4.2: This task will apply the tools developed in WP1 and WP2,
  and the robot simulation environment of the DISCOVER platform, to
  the coordinated multi-robot system, in order to validate the
  implementation code and detect and fix possible bugs.  It will also
  provide feedback to the other work packages to refine and improve
  the tools developed in this project.
\item T4.3: This task will aim to use the work on the robotics effort
  to prototype a mapping from implementation code annotations in
  \acsl/\eacsl and extensions into skeletons of models in timed
  automata formalisms.
  \end{itemize}

As the tasks in this work package are conducted in tandem with WP1 and WP2, to form a feedback loop with the developments in other work packages, it will last for the entire duration of the project.
We expect that groups of undergraduate students, in collaboration with
an embedded systems Ph.D. student and the Ph.D. students in WP1 and WP2, will
perform the work.
Close collaboration with the DISCOVER team, led by Dr. Flikkema and Dr. Nghiem, is expected.

\paragraph{Evaluation:} In essence, this task \emph{is} the evaluation
aspect of our project, which forms one of the major thrusts of the
project.  The successful application of WP1 and WP2 tools to the case
studies is essentially the driving factor in determining our success
in the project.
%, and the key feedback to drive changes to our research
%priorities or technical choices.
The measure of success is: (1) faults detected and corrected; (2)
functionality proven correct using CBMC, symbolic execution engines,
or SPIN; (3) coverage and other measures of generated tests; and (4)
reported usability and value by embedded systems engineers,
particularly students.  For T4.3, the evaluation will be based on a
formal comparison of the extracted skeleton with full timed automata
models developed by embedded systems experts. The degree of success
will be estimated based on the correspondence with a real model.

%\subsubsection{Timeline}
%\label{sec:time-line}


%%% Local Variables:
%%% mode: latex
%%% TeX-master: "main"
%%% End:
