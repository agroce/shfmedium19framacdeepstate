% To deal with space problems:
\usepackage{enumitem}

% Times font
\usepackage{mathptmx}

\usepackage{code}

% Microtype does not work with fontspec:
\usepackage[protrusion=true,expansion=true]{microtype}

\usepackage{doi}
\usepackage{xspace}

%%%% Geometry (paper size, margins, etc.)
% Change the size of body text and margins.  Refer geometry for more
% options.
\usepackage[top=1in,bottom=1in,left=1in,right=1in]{geometry}
%\usepackage[body={6.4in,9in}, top=1in, left=1in]{geometry}
% or simply use fullpage for a full page
% \usepackage[in]{fullpage}

\usepackage{parskip}
\parskip=0mm   % Set the skip size between paragraphs
\parindent=1em

\usepackage{subcaption}

\usepackage{titlesec}
\titlespacing\section{0pt}{8pt plus 4pt minus 4pt}{4pt plus 2pt minus 2pt}
\titlespacing\subsection{0pt}{8pt plus 4pt minus 4pt}{4pt plus 2pt minus 2pt}
\titlespacing\subsubsection{0pt}{8pt plus 4pt minus 4pt}{2pt plus 2pt minus 2pt}
\titlespacing\paragraph{0pt}{6pt plus 4pt minus 4pt}{4pt plus 2pt minus 2pt}

% The standard LaTeX font is Computer Modern. Other fonts are available.
%
% Other popular options are
% - Times:     \usepackage{mathptmx}
% - Palatino:  \usepackage{mathpazo}
% - New Century Schoolbook: \usepackage{newcent}
% - Another (better) Times: \usepackage{ntxtext}  % Times font

%%%% Math support (AMS math packages, symbols...)
% If you want to flush all equations to the left, or to set other
% options of amsmath, use the following line:
% \usepackage[fleqn]{amsmath}
\usepackage{amsmath}
% Uncomment line below to allow page break inside multi-line equations
% \interdisplaylinepenalty=2500

% Additional math packages
\usepackage{amsfonts,amssymb}
\usepackage{bm} % for bold symbols like \bm{\alpha} or {\bm \alpha\beta}
\usepackage{mathrsfs}  % Formal math script font \mathscr
\usepackage{cases}  % Better than amsmath's cases environment

%%%% SI Unit support: should always use to typeset numeric data
% With option per-mode=symbol: kW/s, without it: kW s^{-1}
% Quick usage:
%   \num{1+-2i}
%   \num{.3e45}
%   \SI{number}{unit} or \si{unit}
% where unit is like:
%   kg.m.s^{-1} or \kilogram\metre\per\second
%   \si[per-mode=symbol]{\kilogram\metre\per\ampere\per\second} for
%        kg m / (A s)
\usepackage[per-mode=symbol]{siunitx}


%%%% Graphic packages
\usepackage{graphicx}
\graphicspath{{figs/}}

%\usepackage{subfig}  % Sub-figures
% \usepackage{color}   % Support colors
\usepackage[svgnames, rgb]{xcolor}  % Support for color with standard names

% PGF/TikZ
\usepackage{tikz}
\usepackage{pgfgantt}

\newganttchartelement{bluebar}{
    bluebar/.style={
        inner sep=0pt,
        draw=purple!44!black,
        thick,
        top color=white,
        bottom color=blue!70
    },
    bluebar label font=\slshape,
    bluebar left shift=.1,
    bluebar right shift=-.1
}

%\usetikzlibrary{snakes,arrows,shapes,positioning}
% Some other TikZ libraries
% \usetikzlibrary{automata,calc,matrix}
% \usetikzlibrary{circuits.logic.US,circuits.ee.IEC}
% \usetikzlibrary{shapes.geometric,shapes.symbols,shapes.arrows}

% PGFplots to plot pretty graphs of data or functions
% \usepackage{pgfplots}
% \usepackage{grffile}
% \pgfplotsset{compat=newest}  % Use advanced features
% \usetikzlibrary{plotmarks}
% \usetikzlibrary{arrows.meta}
% \usepgfplotslibrary{patchplots}

% Define lengths for figure sizes, for PGFPlots figures exported from other tools
% \newlength{\figheight}
% \newlength{\figwidth}

\usepackage{wrapfig}  % Wrap text around figures

% \usepackage{vaucanson-g}  % Draw state machines


%%%% Special typesetting
% Algorithms
%\usepackage[ruled]{algorithm}    % For creating floating algorithms, in the algorithms bundle

% The followings are packages for creating actual algorithms
%\usepackage{algorithmic}  % Easy to use, Pascal-like, less flexible, in the algorithms bundle
%\usepackage{algorithm2e}  % More flexible, more like Pascal
%\usepackage{algpseudocode}  % In the algorithmicx package, flexible, more math-like

% For typesetting source code
%\usepackage{listings}

% Notes, comments
% \usepackage{todonotes}
% \usepackage{verbatim}   % better support for verbatims, comment environment

% Review tools: Note, todo, instruction, etc.
\usepackage[tikz]{mdframed}

\global\mdfdefinestyle{instructionbox}{% 
  backgroundcolor=black!10,
  rightline=false,
  topline=false,
  bottomline=false,
  linecolor=black,
  linewidth=2pt}

\global\mdfdefinestyle{todobox}{% 
  backgroundcolor=orange!30,
  rightline=false,
  topline=false,
  bottomline=false,
  linecolor=orange,
  linewidth=2pt}

% Commands to create notes, todos, etc.
\newcommand{\note}[1]{\textcolor{red}{\textbf{#1}}}
\newcommand{\notetruong}[1]{\note{Truong: #1}}
\newcommand{\notealex}[1]{\note{Alex: #1}}
\newcommand{\notefred}[1]{\note{Frederic: #1}}
\newcommand{\notepaul}[1]{\note{Paul: #1}}

\newenvironment{todolist}
{\begin{mdframed}[style=todobox]}
{\end{mdframed}}

\newenvironment{instruction}
{\begin{mdframed}[style=instructionbox]}
{\end{mdframed}}


%%%% Customizations

% Customization of enumerate, itemize, description
%\usepackage{enumerate}  % Customize enumerate only
\usepackage{enumitem}   % Customize enumerate, itemize, description

%\usepackage{cite}  % Fix issues with \cite handling numbers
\usepackage[numbers]{natbib}
\usepackage{array}  % Better tabular and array
\usepackage{booktabs}           % More beautiful tables

%%%% Set-up hyperref for hyperlinks, etc.
%%%% This should always come last in the preamble.

% The default \usepackage{hyperref} should work automatically.
% See hyperref for more options (visual, backlinks, etc.)
% Note that hyperref with pdftex seems fragile, especially with special symbols in section titles.
\usepackage{hyperref}
\usepackage{multicol,listings}
\definecolor{commentcolor}{rgb}{0.7,0.7,0.7}

\lstdefinelanguage{ACSL}{%
  morekeywords={[2]assert,assigns,assumes,axiom,axiomatic,behavior,behaviors,
    boolean,breaks,complete,continues,data,decreases,disjoint,ensures,
    exit_behavior,ghost,global,inductive,integer,invariant,lemma,logic,loop,
    model,predicate,reads,real,requires,sizeof,strong,struct,terminates,type,%returns
    union,variant,writes,\\nothing,\\valid,\\result,\\forall,\\exists,integer,\\separated,\\valid_read,\\list},
  alsoletter={\\},
  morecomment=[l]{//},
  moredelim=*[s][\color{black}]{/*@}{*/}
}


\lstdefinestyle{c}{language={[ANSI]C},%
  alsolanguage=ACSL,%
  commentstyle=\lp@comment,%
  moredelim={*[l]{//}},%
  %deletecomment={[s]{/*}{*/}},
  moredelim={*[l][\color{black}]{//@}},%
}


\lstset{
    escapeinside={\%*}{*)},
    language=C,
    style=c,
    alsolanguage=ACSL,
    basicstyle=\ttfamily,
    commentstyle=\itshape\color{commentcolor},
    keywordstyle={\bfseries\color{black}},
    keywordstyle=[2]{\bfseries\color{black}},
    showstringspaces=false,
    stringstyle={\rmfamily\color{gray}},
    %frame=single,
    columns=flexible,
    aboveskip=6pt,
    belowskip=6pt,
    mathescape=true,
    breaklines=true
}

\lstMakeShortInline"


% Use option [final] to finalize and remove all changes markups
\usepackage{changes}