\section{Related Work}

A fundamental goal of this project is to reduce both user effort and
the opportunity for user effort by allowing minimizing (ideally to
one) the number of times a user must specify an aspect of system
correctness.  The principle that important information should have a
``single point of truth'' is widely accepted in software engineering,
even in such foundational early advances as avoiding repeated magic
numbers by the use of named constants.  Such a principle can be
extended to specification and definition of test harnesses.  Early
work emphasizing this goal of both reducing work and chance of error
in specification and test generation included the effort by Groce and
Joshi to use a single harness for both model-checking and random
testing, in the verification of the Mars Science Laboratory's file
system \cite{WODA08,CFV08,AMAI}.  In later work, Groce and Erwig
extended this idea to propose development of a single language with a
unified semantics for a wide variety of dynamic test generation tools
\cite{WODACommon}; this approach is essentially realized in the
DeepState \cite{DeepState} system.  Indeed, \framac and
\acsl~\cite{ACSL} and DeepState are both arguably limited
instantiations of this goal: providing a single language, interface,
and semantics that is applied to a variety of methods (static or
dynamic) for checking that a specification holds.  This project aims
to further extend this goal by extending it to include a formal
timed-automata model and to connect the primarily static and dynamic approaches.

Verdi~\cite{WWP2015:PLDI,WWA2016:CPP} is a framework for the
implementation and verification of distributed systems using \Coq. The
strength of this framework is to allow users to design and prove the
correctness of a system without considering faults, and apply what is
called a verification system transformer to modify the system into a
system that handles faults.  Systems developed in this way are
executable thanks to \Coq's extraction mechanism (to
OCaml~\cite{MIN2011:CACM} code). Verdi could not be applied to our
case study: time is not considered by Verdi and although it is
possible to port variants of OCaml on
micro-controllers~\cite{VWC2015:PADL}, there is currently no such
variant for the micro-controllers deployed in SEGA. Using a framework
such as Verdi would also mean writing new code using a language
unfamiliar to SEGA's developers. This proposal instead focuses on the
analysis of legacy code.

Testing real-time systems modeled by networks of timed automata was
investigated by the authors of the tool \uppaal
\cite{uppaaltron2008,uppaalcover2007,uppaaltron2005} and implemented
in the tools \uppaal-TRON\footnote{\uppaal-TRON:
  \url{http://people.cs.aau.dk/~marius/tron/index.html}} and
\uppaal-COVER\footnote{\uppaal-COVER:
  \url{http://www.hessel.nu/CoVer/index.php}}.  These tools generate
tests, either offline or online, for comformance testing of a
real-time system with respect to its model and an environment model,
both as timed automaton networks.  In both cases, the real-time system
is considered a black-box with an input/output interface through which
the test generator or monitor can change the system inputs and observe
the system outputs.  The actual implementation code is not considered
and is in fact hidden from the testing tools.  While this approach is
general, it has several drawbacks.  It requires a centralized
input/output interface accessible to the testing tools.  Such an
interface is not always available in all systems, especially in
large-scale distributed systems like the sensor/actuator networks
considered in our case study.  Furthermore, by considering only the
(timed) input/output behavior of a system, this approach may not be
able to test internal system behaviors and therefore miss
opportunities for a better test
coverage.  % and for more scalable testing (by testing internal components of the system instead of the entire system).
Finally, regardless of how thorough the tests are, generally they will
not be able to provide a formal guarantee of the conformance between
the implementation and the model / specifications.  On the contrary,
if the system code is available, our proposed
approach can potentially provide such a
formal guarantee.



% ;;; Local Variables: ***
% ;;; mode: latex ***
% ;;; eval: (ispell-change-dictionary "english" nil) ***
% ;;; eval: (flyspell-buffer) ***
% ;;; End: ***
