\section{Broader Impacts}

\paragraph{Improving Software System Reliability:} A key element of
our approach is to focus on realistically deployable techniques.  We aim
for early integration with NASA's FPrime~\cite{fprime,fprimerepo}
open source
flight software architecture and platform; PI Groce is already in
discussion with engineers at NASA's Jet Propulsion Laboratory, and
engaged in producing tests for the FPrime autocoder using DeepState.
This integration will allow our
methods to be applied to CubeSat missions (and other flight software
systems), leading to improved reliability for low-budget space-based
scientific efforts.  We expect, in the long run, that our approaches
will lead to more reliable and robust development in many embedded and
cyberphysical systems domains.  Our engagement with interested Galois
engineers ensures the applicability of our methods will be as wide as
possible, and impact tools outside our initial scope.

\paragraph{Education and Outreach:}
The proposed research yields several opportunities for enhancing CS
education, recruiting new CS majors, and retaining CS students,
particularly members of underrepresented groups.  In addition to the
activities discussed at length in the Broadening Participation in
Computing plan, PI Groce will work with the NAU Student ACM Chapter to
present a series of ``excursions in testing'' that introduce automated
testing to students, using DeepState to find bugs in real world code,
including code from media player libraries.  The
work of Guzdial \cite{Guzdial} has shown that media computation is an effective way to both recruit and retain female and
underrepresented students in computer science. Groce is also teaching
a class on automated testing of embedded systems.  Co-PI Nghiem has
developed a course on  autonomous vehicles, based on the F1/10 platform (funded by
NSF). 
To prepare students for addressing safety in autonomous driving, future offerings of the course will incorporate the methods and tools developed in this project.
%Each year, Loulergue teaches CS451/551 Mechanized Reasoning about Programs to about 40 students.
%This class is based on \framac, and Loulergue transitions his research on formal methods into this class.
%Loulergue has presented many \framac tutorials in conference such as ISSRE, ACM SAC, FM, SecDev, {\it etc.}, and will continue to do so.


\paragraph{Broadening Participation in Computing (BPC):}
The goal of the BPC component of this project is to \textit{increase the number of females who are involved or choose careers in computing, at NAU and in the local community of Flagstaff, Arizona.}  Our plan carefully integrates active learning experiences designed for female students at both the undergraduate and middle school/junior high levels.
\textbf{Undergraduate Education Experience -} We will reach female students in two degree programs at the 2nd-year level: Computer Science and Electrical and Computer Engineering. In CS, we will target CS 200 Introduction to Computer Organization; in ECE, we will target EE 215 Microprocessors. We will integrate a new project in which teams of female (and possibly male, due to the current lack of females in ECE and CS) students imagine and create exciting and meaningful one-day active learning experiences and projects for female student teams in grades 7-9.  We will provide full support to these teams, especially female students, and design the project so that female students will take leadership roles to gain confidence.  In both courses, we will bring in expert female speakers to facilitate development of students’ understanding how to design these projects so they are marker events in the students’ lives. We will also explicitly address increasing the awareness of the challenges faced by females of all ages in STEM careers.
\textbf{Outreach to grades 7-9 -} As noted above, the undergraduate teams will develop active learning and design project “Build Events” for girls in grades 7-9. We will recruit female undergraduates who have taken CS 200/EE 215 to become mentors in the one-day events for the grade 7-9 students.  We will schedule these events as part of the annual Flagstaff Festival of Science, and plan them for Saturdays to avoid conflicts with school schedules, maximizing participation. The Flagstaff Festival of Science (www.scifest.org), now in its 32nd year and enjoying wide financial and participatory support in the community, holds over 100 events for all ages over a 10-day period in the Fall, and is an ideal venue in which to participate.
\textbf{Facilities and Support -} By scheduling the grade 7-9 Build Events on Saturdays, we will be able to use the educational laboratories of the School of Informatics, Computing \& Cyber Systems (SICCS) for the Flagstaff Festival of Science events. We have requested \$2,000 for each in years 2 and 3 for materials (primarily embedded development boards) for these experiences.
\textbf{Assessment -} We will conduct focused feedback sessions and administer short surveys of the participants to aid continuous improvement of the activities over the course of the project.

%%% Local Variables:
%%% mode: latex
%%% TeX-master: "main"
%%% End:
