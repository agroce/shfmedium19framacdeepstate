\section{Broader Impacts}

\paragraph{Improving Software System Reliability:} A key element of
our approach is to focus on realistically deployable techniques.  Part
of this effort is concentrated in our internal effort to apply our
methods to the SEGA project.  However, we also plan
to aim for early integration with NASA's FPrime~\cite{fprime,fprimerepo}
open source
flight software architecture and platform; PI Groce is already in
discussion with engineers at NASA's Jet Propulsion Laboratory, and
engaged in producing tests for the FPrime autocoder using DeepState.
This integration will allow our
methods to be applied to CubeSat missions (and other flight software
systems), leading to improved reliability for low-budget space-based
scientific efforts.  We expect, in the long run, that our approaches
will lead to more reliable and robust development in many embedded and
cyberphysical systems domains, and contribute to a more secure and
reliable Internet of Things.  One key goal of this project is to
increase the synergy between formal modeling, heavyweight static
analysis, and advanced dynamic analysis using automated test
generation tools, and thus adoption of all three methods.

\paragraph{Education and Outreach:}
The proposed research yields several opportunities for enhancing CS
education, recruiting new CS majors, and retaining CS students,
particularly members of underrepresented groups.  In addition to the
activities discussed at length in the Broadening Participation in Computing plan,
PI Groce will work with the NAU Student ACM Chapter to present a
series of ``excursions in testing'' that introduce automated testing
to students, using DeepState to find bugs in real world code, including code from
media player libraries; in advanced meetings, integrating DeepState
with \framac will be demonstrated as well.  The work of Guzdial
\cite{Guzdial} has shown that media computation is a
potentially effective way to both recruit and retain female and
under-represented minority students in computer science. Groce is also teaching a
class on automated testing of embedded systems to graduate and
undergraduate students.
Co-PI Nghiem is preparing a new graduate-level course on autonomous
vehicles, based on the F1/10 platform (\url{http://f1tenth.org/}), to be offered to EE and CS students at NAU.
To prepare students in addressing one of the greatest challenges of autonomous driving, namely safety guarantee, the course will incorporate the methods and tools developed in this project to teach students about safety, verification, and testing.
Each year, Loulergue teaches CS451/551 Mechanized Reasoning about Programs to about 40 students.
This class is based on \framac, and Loulergue transitions his research on formal methods into this class.
Loulergue has presented many \framac tutorials in conference such as ISSRE, ACM SAC, FM, SecDev, {\it etc.}, and will continue to do so.

%%% Local Variables:
%%% mode: latex
%%% TeX-master: "main"
%%% End:
