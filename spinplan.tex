Just as some functions are best analyed using bounded model checking, some dynamic analysis problems are best handled by explicit-state model checking that actually executes C code, like a fuzzer, but with the capability to store states and backtrack, in order to exhaustively explore a state space, using either actual comparison of stored states or comparison of abstractions of states to guide exploration.  This approach is particularly attractive for exploring sequences of API calls; this kind of test generation was used in efforts that uncovered dozens of errors in file systems at NASA/JPL~\cite{AMAI} and in a widely-used mock file system for Python testing used by over 1,000 projects~\cite{tstl}.

The SPIN model checker~\cite{SPIN} offers execution of C code with backtracking~\cite{ModelDriven,ModelCode}.  DeepState's {\tt OneOf} construct has a semantics that can be matched with the SPIN nondeterministic choice, which in part inspired the DeepState construct~\cite{WODA08,WODACommon}.  However, integrating SPIN as a back-end for DeepState is even more challenging than integrating CBMC.  With CBMC, it is plausible that the mapping from DeepState to CBMC semantics can be performed entirely in terms of changing included headers so that CBMC-specific constructs have differing implementations (but not semantics); SPIN however executed C code in the context of a PROMELA model, which requires rewriting a DeepState model to embed test choices inside SPIN's constructs.  This also means ``lifting'' DeepState API calls to the PROMELA level outside the C code, and bridging between nondeterminism visible to SPIN and determinism within C code.  Such a translation is not the core research challenge however, and a substantial elaboration of techniques develop by PI Groce as long as as 2008 at JPL~cite{WODA08} can serve as a foundation for solving this problem.  The more fundamental problem is that while CBMC and DeepState can share a semantics for, e.g., {\tt DeepState\_Int()}, allowing all values of the appropriate bit-size, a PROMELA model with a branching factor of, e.g., $2^{64}$ is completely useless (and, indeed, will not work).  There are multiple possible solutions, ranging from using results from fuzzing to choose a limited range, to translating ``flat'' bit-value selection into a sequence of choices with a strong bias to a set of initial values, but an in-principle unlimited range, to using SPIN to control a seed and deterministically choosing random values, a hybrid fuzzing/explicit-state model checking approach.  All of these approaches have potential problems (e.g., sequences that construct numeric values cause significant state-space explosion), but all may be needed to handle different embedded systems verification problems.